
\documentclass{article}
\usepackage{colm2024_conference}
\usepackage{microtype}
\usepackage{hyperref}
\usepackage{url}
\usepackage{booktabs}
\usepackage{amsmath}
\usepackage{amssymb}
\usepackage{graphicx}
\usepackage{amsthm}
\usepackage{bm}

\definecolor{darkblue}{rgb}{0, 0, 0.5}
\hypersetup{colorlinks=true, citecolor=darkblue, linkcolor=darkblue, urlcolor=darkblue}

\newtheorem{theorem}{Theorem}

\title{Extending Context Length in Attention Mechanisms with Fourier Features}

\author{Anonymous Author(s)}

\begin{document}

\maketitle

\begin{abstract}
This paper explores the use of Fourier features to extend the context length in attention mechanisms within deep learning models. By leveraging the properties of Fourier transforms, we provide a framework for understanding and implementing context length extension through linear interpolation and change of base techniques. This approach offers insights into model behavior and guides the design of more effective architectures.
\end{abstract}

\section{Introduction}

Fourier features have shown promise in improving model performance, particularly in the context of extending the effective context length in attention mechanisms. This paper explores how Fourier features can be utilized to achieve context length extension, revealing a perspective that enhances our understanding of modern deep learning models.



\subsection{Background: Fourier Features in Attention Mechanisms}

The challenge of modeling long contexts in Transformers is primarily due to the limitations of the attention mechanism. Attention operates over $C$ embeddings $\mathbf{X} = [\mathbf{x}_1, \mathbf{x}_2, \ldots, \mathbf{x}_C]^\top \in \mathbb{R}^{C \times d}$, where $d$ is the model dimension. Fourier features offer a powerful tool for enhancing these mechanisms by encoding positional information through frequency-based transformations. Learned weight matrices $\mathbf{W}_v \in \mathbb{R}^{d \times d_k}$, $\mathbf{W}_q \in \mathbb{R}^{d \times d_k}$, and $\mathbf{W}_k \in \mathbb{R}^{d \times d_k}$ are used to transform these inputs, where $d_k$ is the projected hidden dimension. The attention mechanism computes the attention matrix and applies it to produce a weighted sum of the value vectors:
\begin{equation}
\text{Attention}(\mathbf{Q}, \mathbf{K}, \mathbf{V}) = \mathbf{A} \mathbf{V} = \text{softmax}\left(\frac{\mathbf{Q} \mathbf{K}^\top}{\sqrt{d_k}}\right)\mathbf{V}.
\end{equation}
Basic attention was originally defined with: $\mathbf{Q} = \mathbf{X} \mathbf{W}_q, \mathbf{K} = \mathbf{X} \mathbf{W}_k, \mathbf{V} = \mathbf{X} \mathbf{W}_v$. However, this approach does not directly encode the relative position of keys and values, which is where Fourier features come into play. By leveraging the properties of Fourier transforms, we can encode positional information more effectively.

Rotary Position Embeddings (RoPE) \citep{su2024roformer} utilize Fourier features by applying a phase rotation to each element of the embedding vectors. Formally, we define a general transformation $\mathbf{f}$:


\begin{equation}
\mathbf{f}_\mathbf{W}(\mathbf{x}_i, \bm{\theta}) = \mathbf{R}(\bm{\theta}, i)\mathbf{W}^\top \mathbf{x}_i
\end{equation}

Here $\mathbf{x}_i \in \mathbb{R}^{d_k}$ is an embedding for position $i$, $\mathbf{W}$ is a projection matrix, and $\bm{\theta} \in \mathbb{R}^{d_k / 2}$ is a frequency basis. The function is defined based on the rotary position matrix: 

\begin{equation}
\mathbf{R}(\bm{\theta},i)= \begin{pmatrix}
\cos i\theta_1 & - \sin i\theta_1 &  \cdots & 0 & 0 \\
\sin i\theta_1 & \cos i\theta_1 & \cdots & 0 & 0 \\
% 0 & 0 & \cos m\theta_2 & - \sin m\theta_2 & \cdots & 0 & 0 \\
% 0 & 0 & \sin m\theta_2 & \cos m\theta_2 & \cdots & 0 & 0 \\
\vdots \\ 
0 & 0 &  \cdots & \cos i\theta_\frac{d_k}{2}  & - \sin i\theta_\frac{d_k}{2}  \\
0 & 0 &  \cdots & \sin i\theta_\frac{d_k}{2}  & \cos i\theta_\frac{d_k}{2}  \\
\end{pmatrix}
\end{equation}





% \begin{equation}
% (\mathbf{R}_{\theta,m})_j = 
% \begin{bmatrix}
% \text{cos} m\theta_j & -\text{sin} m\theta_j \\
% \text{sin} m\theta_j & \text{cos} m\theta_j
% \end{bmatrix}.
% \end{equation}
 
% For example, $\mathbf{q_m}$ can be viewed as the following, 

% \[
% \mathbf{f}_{\mathbf{W}_q}(\mathbf{x}_m, m) = 
% \begin{pmatrix}
% \text{cos} m\theta_1 & - \text{sin} m\theta_1 & 0 & 0 & \cdots & 0 & 0 \\
% \text{sin} m\theta_1 & \text{cos} m\theta_1 & 0 & 0 & \cdots & 0 & 0 \\
% 0 & 0 & \text{cos} m\theta_2 & - \text{sin} m\theta_2 & \cdots & 0 & 0 \\
% 0 & 0 & \text{sin} m\theta_2 & \text{cos} m\theta_2 & \cdots & 0 & 0 \\
% 0 & 0 & 0 & 0 & \cdots & \text{cos} m\theta_\frac{d}{2}  & - \text{sin} m\theta_\frac{d}{2}  \\
% 0 & 0 & 0 & 0 & \cdots & \text{sin} m\theta_\frac{d_k}{2}  & \text{cos} m\theta_\frac{d_k}{2}  \\
% \end{pmatrix}\mathbf{W^\top_q}\mathbf{x}_m.
% \]

Due to the arrangement of frequencies, this matrix has the property that $\mathbf{R}(\bm{\theta},n - m) = \mathbf{R}(\bm{\theta},m)^\top\mathbf{R}(\bm{\theta},n)$. We redefine the query-key product between two positions $m$ and $n$ as, 

\begin{align}
\mathbf{q}^\top_m{\mathbf{k}_n}&=\mathbf{f}_{\mathbf{W}_q}(\mathbf{x}_m, m, \bm{\theta})^\top\mathbf{f}_{\mathbf{W}_k}(\mathbf{x}_n, n, \bm{\theta}) \\ 
& = \left(\mathbf{R}({\bm{\theta},m})\mathbf{W}^\top_q\mathbf{x}_m \right)^\top \left( \mathbf{R}(\bm{\theta},n)\mathbf{W}^\top_k\mathbf{x}_n \right) \\
&= \mathbf{x}_m^\top\mathbf{W}_q \mathbf{R}(\bm{\theta},n - m) \mathbf{W}_k^\top\mathbf{x}_n 
\end{align}

In this way, the relative positional information $n-m$
is implicitly injected into the query and key product, thus the attention score.

The standard RoPE transformation utilizes: 

\begin{equation}
\mathbf{f}_\mathbf{W}^{\text{RoPE}}(x_i) = \mathbf{f}(x_i, \bm{\theta}^d).
\end{equation}

where $\bm{\theta}^d_j = b^{-\frac{2j}{d_k}}$ and base $b=10000$. 



% We can then obtain RoPE-based attention at $m$-th position as 

% \begin{equation}
% \text{Attention}(\mathbf{Q}, \mathbf{K}, \mathbf{V})_m = \frac{\sum_{n=1}^{C}\exp(\mathbf{q}_m^T \mathbf{k}_n / \sqrt{d_k})\mathbf{v}_n}{\sum_{n=1}^{C} \exp(\mathbf{q}_m^T \mathbf{k}_n / \sqrt{d_k})}.
% \end{equation}

% where $\mathbf{v}_n = \mathbf{R}_{\bm{\theta},n}\mathbf{W}^\top_v\mathbf{x}_n$. Note that this calculation still yields $O(C^2)$ complexity.

\subsection{Adjusting Frequency Base of RoPE for Long Context Extensions}

Recent papers focus on modifying the RoPE equation to extend length. We consider four methods: Position Interpolation (PI)~\citep{chen2023extending}, NTK-RoPE~\citep{emozillareddit}, YaRN~\citep{peng2023yarn} and CLEX~\citep{chen2024clexcontinuouslengthextrapolation}. In this section we assume that the goal is to extend a method trained to work with $C$ positions to instead work with $C' >> C$. We refer to the ratio $\alpha = \frac{C'}{C}$ as the \textit{scale factor}.

\textbf{Linear Position Interpolation (PI)} involves down-scaling the frequency coefficient $i$ to align with the original context size. PI has been integrated into many open-source models such as LLaMA-2-7B-32K~\citep{together-instruct}, Vicuna-7B-v1.5~\citep{vicuna2023}, and LongAlpaca~\citep{chen2023longlora}. PI works by interpolating the position for a token originally at position $i$ by a scale factor to be at pseudo-position $\frac{i}{\alpha}$.
 PI can be implemented by changing the frequency base $\bm{\theta}$:
\begin{equation}
\mathbf{f}_\mathbf{W}^{\text{PI}}(x_i) = \mathbf{f}(x_i, \alpha^{-1} \odot \bm{\theta}^d).
\end{equation}


\textbf{Neural Tangent Kernel Interpolation RoPE (NTK-RoPE) } modifies the base frequency of the rotation in RoPE by modifying $\bm{\theta}^d_j = b^{-\frac{2j}{d_k}}$ to $\bm{\theta}^d_j = (b\kappa)^{-\frac{2j}{d_k}}$.  When $j= \frac{d_k}{2}-1$, NTK-RoPE sets  $(b\kappa)^{-\frac{2j}{d_k}} = \frac{1}{\alpha} (b^{-\frac{2j}{d_k}})$.  Hence $\kappa$ can be derived as  $\kappa = (\frac{C'}{C})^{\frac{d_k}{d_k-2}} = \alpha^{\frac{d_k}{d_k-2}}$. NTK-RoPE can be written as 

\begin{equation}
 \mathbf{f}_\mathbf{W}^{\text{NTK-RoPE}}(\mathbf{x}_i) = \mathbf{f}(\mathbf{x}_i,   \alpha^{-\frac{2j}{d_k-2}} \odot \bm{\theta}^d),
\end{equation}

An extension to this approach, Dynamic NTK-RoPE suggests that instead of using a fixed scale factor $\alpha$ during inference, the formula should adapt to the current context length for a specific example. One implementation updates the scale factor to $\alpha_{\text{test}} = \frac{s C_\text{test}}{C} - (s - 1)$, where $s \in [1, \frac{C'}{C}]$ is a hyperparameter used to adjust $\alpha_{\text{test}}$ and $C_\text{test}$ is maximum observed length during training or inference ~\citep{fu2024data}.

Another approach from \citep{fu2024data}, sets $\alpha_{\text{test}} = s \cdot \frac{\max(C', C_\text{test})}{C}-(s-1)$, where $s$ is set to $\frac{C'}{2C}$ during training and inference.

\bibliography{colm2024_conference}
\bibliographystyle{colm2024_conference}

\end{document}
