
\documentclass{article}
\usepackage{colm2024_conference}
\usepackage{microtype}
\usepackage{hyperref}
\usepackage{url}
\usepackage{booktabs}
\usepackage{amsmath}
\usepackage{amssymb}
\usepackage{graphicx}

\definecolor{darkblue}{rgb}{0, 0, 0.5}
\hypersetup{colorlinks=true, citecolor=darkblue, linkcolor=darkblue, urlcolor=darkblue}

\title{Unifying Position Embeddings, Neural Tangent Kernel, and Fourier Features}

\author{Anonymous Author(s)}

\begin{document}

\maketitle

\begin{abstract}
This paper explores the intricate connections between position embeddings, the Neural Tangent Kernel (NTK), and Fourier features in deep learning models. We demonstrate how these concepts are fundamentally linked through the lens of harmonic analysis and kernel methods. Our analysis reveals that position embeddings can be viewed as approximations of the NTK, which in turn can be expressed using Fourier features. We show how Euler's formula provides a unifying framework for understanding these relationships, offering insights into model behavior and guiding the design of more effective architectures.
\end{abstract}

\section{Introduction}

Position embeddings have become an essential component in many deep learning architectures, particularly in sequence modeling tasks. Concurrently, the Neural Tangent Kernel (NTK) has emerged as a powerful tool for analyzing neural network behavior, while Fourier features have shown promise in improving model performance. This paper explores the deep connections between these concepts, revealing a unifying perspective that enhances our understanding of modern deep learning models.

\section{Theoretical Foundations}

\subsection{Position Embeddings and Fourier Features}

Position embeddings, particularly sinusoidal embeddings used in Transformer models \citep{Vaswani2017}, can be viewed as a form of Fourier feature mapping. Consider the sinusoidal position embedding function $\phi(x)$ for a 1D input $x$:

\begin{equation}
    \phi(x) = [\sin(\omega_1 x), \cos(\omega_1 x), \sin(\omega_2 x), \cos(\omega_2 x), ..., \sin(\omega_d x), \cos(\omega_d x)]
\end{equation}

where $\omega_i = 1/10000^{2i/d}$ for $i = 0, 1, ..., d/2 - 1$.

This embedding is essentially a Fourier feature mapping with a specific choice of frequencies. The connection becomes even clearer when we consider Euler's formula:

\begin{equation}
    e^{i\omega x} = \cos(\omega x) + i\sin(\omega x)
\end{equation}

Using Euler's formula, we can express the position embedding in complex form:

\begin{equation}
    \phi(x) = [e^{i\omega_1 x}, e^{-i\omega_1 x}, e^{i\omega_2 x}, e^{-i\omega_2 x}, ..., e^{i\omega_d x}, e^{-i\omega_d x}]
\end{equation}

This representation highlights the connection to the Fourier transform and provides a bridge to the Neural Tangent Kernel.

\subsection{Neural Tangent Kernel and Fourier Features}

The Neural Tangent Kernel is a theoretical framework that describes the behavior of wide neural networks during training \citep{Jacot2018}. Interestingly, the NTK can be expressed in terms of Fourier features.

For a neural network with activation function $\sigma$, the NTK can be written as:

\begin{equation}
    K(x, x') = \int_{\mathbb{R}} \hat{\sigma}(\omega) \hat{\sigma}(\omega) e^{i\omega(x-x')} d\omega
\end{equation}

where $\hat{\sigma}$ is the Fourier transform of $\sigma$. This formulation reveals that the NTK is essentially a convolution in the frequency domain, which can be approximated using Fourier features.

\subsection{Unifying Perspective}

The connections between position embeddings, NTK, and Fourier features can be summarized as follows:

\begin{enumerate}
    \item Position embeddings, when viewed through Euler's formula, are a form of Fourier feature mapping.
    \item The NTK can be expressed as an integral over Fourier features.
    \item Position embeddings can be seen as an approximation of the NTK for certain activation functions.
\end{enumerate}

This unifying perspective provides several insights:

\begin{enumerate}
    \item The effectiveness of position embeddings can be partially explained by their ability to approximate the NTK.
    \item The choice of frequencies in position embeddings can be guided by analysis of the NTK in the Fourier domain.
    \item The connection to Fourier features explains why position embeddings help models capture periodic patterns and overcome spectral bias.
\end{enumerate}

\section{Practical Implications}

Understanding the connections between position embeddings, NTK, and Fourier features has several practical implications:

\begin{enumerate}
    \item Architecture design: The frequency spectrum of the NTK can guide the design of more effective position embedding schemes.
    \item Initialization strategies: Insights from the NTK can inform better initialization of position embeddings for improved training dynamics.
    \item Transfer learning: The connection to Fourier features suggests that models with well-designed position embeddings may transfer more effectively across tasks with different input distributions.
    \item Interpretability: Analyzing position embeddings in the Fourier domain can provide insights into what patterns the model has learned to recognize.
\end{enumerate}

\section{Conclusion}

By exploring the connections between position embeddings, the Neural Tangent Kernel, and Fourier features through the lens of Euler's formula, we gain a deeper understanding of the theoretical underpinnings of modern deep learning architectures. This unifying perspective not only enhances our theoretical understanding but also provides practical guidance for designing more effective and interpretable models. Future work may further explore these connections to develop novel architectures and training strategies that leverage these insights.

\bibliography{colm2024_conference}
\bibliographystyle{colm2024_conference}

\end{document}
